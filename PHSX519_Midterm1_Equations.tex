% --------------------------------------------------------------
% This is all preamble stuff that you don't have to worry about.
% Head down to where it says "Start here"
% --------------------------------------------------------------
 
\documentclass[10pt]{article}
 
\usepackage[margin=.60in]{geometry} 
\usepackage{amsmath,amsthm,amssymb, mathtools}
\usepackage{multicol}
\usepackage[subnum]{cases}
\usepackage{relsize}
\usepackage[makeroom]{cancel}
\usepackage[english]{babel}
\usepackage{graphicx}
\usepackage{calligra}
\usepackage[normalem]{ulem}
\usepackage{caption}
\usepackage{subcaption}
\usepackage{fancyhdr}

\DeclareMathAlphabet{\mathcalligra}{T1}{calligra}{m}{n} 
\DeclareFontShape{T1}{calligra}{m}{n}{<->s*[2.2]callig15}{}


% Makes '\sr' make a script r
\newcommand{\sr}{\ensuremath{\mathcalligra{r}}}
 
\newcommand{\N}{\mathbb{N}}
\newcommand{\Z}{\mathbb{Z}}
\newcommand{\ihat}{\boldsymbol{\hat{\textbf{\i}}}}
\newcommand{\jhat}{\boldsymbol{\hat{\textbf{\j}}}}
\newcommand{\khat}{\boldsymbol{\hat{\textbf{k}}}}
\newcommand{\rhat}{\boldsymbol{\hat{\textbf{r}}}}
\newcommand{\srhat}{\boldsymbol{\hat{\textbf{\sr}}}}
\newcommand{\xhat}{\boldsymbol{\hat{\textbf{x}}}}
\newcommand{\yhat}{\boldsymbol{\hat{\textbf{y}}}}
\newcommand{\zhat}{\boldsymbol{\hat{\textbf{z}}}}
\newcommand{\nhat}{\boldsymbol{\hat{\textbf{n}}}}
\newcommand{\phihat}{\boldsymbol{\hat{\textbf{$\phi$}}}}

\newcommand{\vect}[1]{\mathbf{#1}}
\newcommand{\vc}[1]{\mathbf{#1}}
\newcommand{\fracl}[2]{\mathlarger{\frac{#1}{#2}}}
\newcommand{\dd}{\, \mathrm{d}}
\newcommand{\eo}{\epsilon_0}
\newcommand{\mo}{\mu_\circ}
\newcommand{\tder}[2]{\frac{\dd #1}{\dd #2}}
\newcommand{\pder}[2]{\frac{\partial #1}{\partial #2}}
\newcommand{\dtder}[2]{\frac{\dd^2 #1}{\dd #2^2}}
\newcommand{\dpder}[2]{\frac{\partial^2 #1}{\partial #2^2}}
\newcommand{\intas}{ \int_{-\infty}^\infty}
\newcommand{\wt}[1]{\widetilde{#1}}
\newcommand{\ev}[1]{\left\langle #1 \right\rangle}
\newcommand{\ce}{\wt{\vect{E}}}
\newcommand{\cb}{\wt{\vect{B}}}
\newcommand{\K}{\frac{1}{4 \pi \eo}}
 
\newenvironment{theorem}[2][Theorem]{\begin{trivlist}
\item[\hskip \labelsep {\bfseries #1}\hskip \labelsep {\bfseries #2.}]}{\end{trivlist}}
\newenvironment{lemma}[2][Lemma]{\begin{trivlist}
\item[\hskip \labelsep {\bfseries #1}\hskip \labelsep {\bfseries #2.}]}{\end{trivlist}}
\newenvironment{exercise}[2][Exercise]{\begin{trivlist}
\item[\hskip \labelsep {\bfseries #1}\hskip \labelsep {\bfseries #2.}]}{\end{trivlist}}
\newenvironment{problem}[2][Problem]{\begin{trivlist}
\item[\hskip \labelsep {\bfseries #1}\hskip \labelsep {\bfseries #2.}]}{\end{trivlist}}
\newenvironment{question}[2][Question]{\begin{trivlist}
\item[\hskip \labelsep {\bfseries #1}\hskip \labelsep {\bfseries #2.}]}{\end{trivlist}}
\newenvironment{corollary}[2][Corollary]{\begin{trivlist}
\item[\hskip \labelsep {\bfseries #1}\hskip \labelsep {\bfseries #2.}]}{\end{trivlist}}


\newenvironment{Figure}
  {\par\medskip\noindent\minipage{\linewidth}}
  {\endminipage\par\medskip}

\pagestyle{fancy}
\lhead{Midterm 1 Equations}
\chead{PHSX519 Electromagnetic Theory I}
\rhead{Roy Smart}

 
\begin{document}
\setlength{\abovedisplayskip}{0pt}
\setlength{\belowdisplayskip}{0pt}
\setlength{\abovedisplayshortskip}{0pt}
\setlength{\belowdisplayshortskip}{0pt}
\begin{align*}
& \vect{E}(\vect{x}) = \frac{1}{4 \pi \eo} \int \rho(\vect{x}') \frac{\vect{x} - \vect{x}'}{|\vect{x} - \vect{x}'|^3} d^3x' \tag*{Coulomb's Law (1.5)} \\
& \delta(f(x)) = \sum_{i} \frac{1}{\left|\tder{f}{x}(x_i)\right|}\delta(x - x_i)	\tag*{Delta function Rule 5 } \\
& \oint_S \vc{E} \cdot \vc{n} \; da = \frac{1}{\eo} \int_V \rho(\vc{x}) d^3x	\tag*{Gauss' Law (1.11)} \\
& \vc{\nabla} \times \vc{E} = 0	\tag*{Curl of electric field (1.14)} \\
& \vc{E} = -\vc{\nabla} \Phi	\tag*{Electric field in terms of scalar potential (1.16)} \\
& \Phi(\vc{x}) = \K \int \frac{\rho (\vc{x}')}{|\vc{x} - \vc{x}'|} d^3 x' \tag*{Scalar potential in terms of charge density (1.17)} \\
& (\vc{E_2} - \vc{E_1}) \cdot \vc{n} = \sigma/\eo	\tag*{Electric field of a surface distribution (1.22)} \\
& \nabla^2 \Phi = -\rho/\eo	\tag*{Poisson Equation (1.28)}\\
& \nabla^2 \Phi = 0		\tag*{Laplace Equation (1.29)}\\
& \nabla^2 \left( \frac{1}{|\vc{x} - \vc{x}'|} \right) = - 4 \pi \delta(\vc{x} - \vc{x}') \tag*{Laplace's equation for a point charge (1.31)} \\
& G(\vc{x}, \vc{x}') = \frac{1}{|\vc{x} - \vc{x}'|} + F(\vc{x}, \vc{x}')	\tag*{Green's function for Poisson's equation (1.40)} \\
& \Phi(\vc{x}) = \K \int_V \rho(\vc{x}') G_D(\vc{x}, \vc{x}') d^3x' = \frac{1}{4 \pi} \oint_S(\vc{x}') \pder{G_D}{n'} da'	\tag*{Green's function potential Dirichlet B.C.s (1.44)} \\
& \Phi(\vc{x}) = \langle\Phi\rangle_S + \K \int_V \rho(\vc{x}') G_N(\vc{x},\vc{x}') d^3x + \frac{1}{4 \pi} \oint_S \pder{\Phi}{n'} G_N da'	\tag*{Green's function potential Neumann B.C.s (1.46)} \\
& W = \frac{\eo}{2} \int |\vc{\nabla} \Phi|^2 d^3x = \frac{\eo}{2} \int \left| \vc{E} \right|^2 d^3 x	\tag*{Energy stored in electric field (1.54)} \\
& V_i = \sum_{j=1}^{n} p_{ij} q_j, \; (i = 1,2,...,n)	
\qquad Q_i = \sum_{j=1}^{n} C_{ij} V_j, \; (i=1,2,...,n) \tag*{Capactiance matrix (1.62)} \\
& W = \frac{1}{2} \sum_{i=1}^{n} Q_i V_i = \frac{1}{2} \sum_{i=1}^n \sum_{j=1}^{n} C_{ij} V_i V_j	\tag*{Potential energy of conductor system (1.62)} \\
& q' = -\frac{a}{y} q, \quad y' = \frac{a^2}{y}		\tag*{Magnitude and position of image charge on sphere (2.4)} \\
& \Phi = -E_0 \left( r - \frac{a^3}{r^2} \right)	\tag*{Scalar potential of conducting sphere in uniform Electric field (2.14)} \\
& \frac{1}{|\vc{x} - \vc{x}'|} = 4 \pi \sum_{\ell = 0}^\infty \sum_{m = -\ell}^{\ell} \frac{1}{2 \ell +1} \frac{r_<^\ell}{r_>^{\ell +1}} Y_{\ell m}^*(\theta', \phi') Y_{\ell m}(\theta,\phi)	\tag*{Green's function expansion in spherical coordinates (3.70)} \\
& \text{Where $r_< (r_>)$ is the smaller (larger) of $|\vc{x}|$ and $|\vc{x}'|$}
\end{align*}

{\renewcommand{\arraystretch}{1.5}
\begin{tabular}{| l | c | c |} \hline
& Wiggly & Decaying \\ \hline
Cartesian &$ e^{\pm i k_n x}, \; A \cos(k_n x) + B\sin(k_n x)$ & $e^{\pm k_n x}, \; A \cosh( k_n x) + B \sinh(k_n x)$ \\ \hline
Cylindrical (3D)& $e^{i m \phi}, \; A J_m(k_n \rho) + B Y_m(k_n \rho)$ & $ A I_m(k_n \rho) + B K_m(k_n \rho)$ \\ \hline
Cylindrical (2D)& $e^{i m \phi}$ & $A_0 + B_0 \ln \rho + \sum A_m \rho^m + B_m \rho^{-m}$ \\ \hline
Spherical ($\phi$ symmetry) & $P_\ell(\cos \theta)$ & $A \left( \frac{r}{a} \right)^\ell + B \big( \frac{r}{a} \big)^{-(\ell+1)} $ \\ \hline
Spherical (No $\phi$ symmetry) & $Y_{\ell m}(\theta, \phi)$ &  $A \left( \frac{r}{a} \right)^\ell + B \big( \frac{r}{a} \big)^{-(\ell+1)} $ \\ \hline
\end{tabular}

\begin{align*}
& \int_{0}^{L} \sin \left( \frac{n \pi x}{L} \right)  \sin \left( \frac{m \pi x}{L} \right) dx = \frac{L}{2} \delta_{nm} 
\qquad \int_{0}^{L} \sin \left( \frac{n \pi x}{L} \right)  \cos \left( \frac{m \pi x}{L} \right) dx = \frac{L n [1 + \cos(n \pi)]}{\pi (n^2 -1)} \\
& \int_{0}^{2 \pi} d \phi \int_{0}^{L} dz e^{i(m - m') \phi} e^{i \frac{2 \pi}{L}(n -n')z} = 2 \pi L \delta_{m'm} \delta_{n'n} \\
& J_m(k \rho) \propto (k \rho)^m \quad Y_m(k \rho) \propto (k \rho)^{-m} \quad I_m(k \rho) \propto (k \rho)^m \quad K_m(k \rho) \propto (k \rho)^{-m} \tag*{As their argument approaches zero}
\end{align*}


% --------------------------------------------------------------
%     You don't have to mess with anything below this line.
% --------------------------------------------------------------
 
\end{document}
